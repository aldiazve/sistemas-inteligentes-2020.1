\begin{frame}{Estrategias de búsqueda}
Una estrategia es definida como escoger un  \textcolor{red}{orden de los nodos de expansión.} \newline
Las estrategias son evaluadas a partir de las siguientes parametros:\newline
\blank{1cm}\textcolor{blue}{completitud} $-$ si existe ¿siempre encuentra la solución? \newline
\blank{1cm}\textcolor{blue}{complejidad temporal} $-$ número de nodos generados/expandidos\newline
\blank{1cm}\textcolor{blue}{complejidad espacial} $-$ máximo número de nodos en memoria.\newline
\blank{1cm}\textcolor{blue}{óptimo} $-$ ¿siempre encuentra la solución menos costosa?\newline
las complejidades temporales y espaciales se miden en términos de\newline
\blank{1cm}\textcolor{pink}{$b$} $-$ actor máximo de ramificación en el árbol de búsqueda
\newline
\blank{1cm}\textcolor{pink}{$d$}$-$ máximo número de nodos en memoria.\newline
\blank{1cm}\textcolor{pink}{$m$}$-$ profundidad máxima de los estados en el espacio (puede ser $\infty$).\newline
\end{frame}
