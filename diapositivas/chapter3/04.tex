% -------------------------- Chapter 3 - Slide 4 -------------------------- % 

\begin{frame}{Agente de resolución de problemas}
Forma restringida de un agente general:

\begin{theorem}[Pseudocódigo]

\begin{listing}
{\scriptsize 
\textcolor{darkblue}{función} \textcolor{darkred}{AGENTE-SOLUCIONADOR-DE-PROBLEMAS-SIMPLE}(\textcolor{darkgreen}{percepción}) 
\textcolor{darkblue}{retorna} una acción
\\* \quad\quad
\textcolor{darkblue}{estático}: 
\textcolor{darkgreen}{sec}, una secuencia de acciones, inicialmente vacía
\\* \quad\quad\quad\quad\quad\quad
\textcolor{darkgreen}{estado}, una descripción del estado actual del ambiente
\\* \quad\quad\quad\quad\quad\quad
\textcolor{darkgreen}{objetivo}, un objetivo, inicialmente nulo
		  problema, una formulación de un problema
\\* \quad\quad
\textcolor{darkgreen}{estado}$\ \leftarrow $ {ACTUALIZAR-ESTADO}(
\textcolor{darkgreen}{estado}, 
\textcolor{darkgreen}{percepción})
\\*\quad\quad
\textcolor{darkblue}{si} 
\textcolor{darkgreen}{sec} está vacío 
\textcolor{darkblue}{entonces} 
\\*\quad\quad\quad\quad
\textcolor{darkgreen}{objetivo}$\ \leftarrow $ Formular-Estado(
\textcolor{darkgreen}{estado}, 
\textcolor{darkgreen}{percepción})
\\*\quad\quad\quad\quad
\textcolor{darkgreen}{problema} $\ \leftarrow $ Formular-Problema(
\textcolor{darkgreen}{estado}, 
\textcolor{darkgreen}{objetivo})\\*
\quad\quad\quad\quad
\textcolor{darkgreen}{sec} $\ \leftarrow $ Buscar(
\textcolor{darkgreen}{problema})\\*
\quad\quad
\textcolor{darkgreen}{acción} $\ \leftarrow $ Recomendación(
\textcolor{darkgreen}{sec}, 
\textcolor{darkgreen}{estado})\\*
\quad\quad
\textcolor{darkgreen}{sec} $\ \leftarrow $Recordatorio(
\textcolor{darkgreen}{sec}, 
\textcolor{darkgreen}{estado})\\*
\quad\quad
\textcolor{darkblue}{retorna} acción
}
\end{listing}

\end{theorem}

{\tiny 
\quad\quad\quad Nota: Esto es resolución de problemas de manera "\textcolor{darkgreen}{offline}"; solución ejecutada con "ojos cerrados".
\\*\quad\quad\quad
La solución de problemas 
"\textcolor{darkgreen}{online}" requiere actuar con conocimiento incompleto.}

\end{frame}
