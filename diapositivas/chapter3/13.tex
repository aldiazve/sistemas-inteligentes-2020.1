%Capitulo 3 Pagina 13
\begin{frame}{Seleccionando un Espacio de estados}{Cap 3 p 13}
    
    El mundo real es absurdamente complejo \newline
    \hspace*{1em} $\Rightarrow$ El espacio de estados debe \textcolor{red}{abstraerse} para resolver el problema \newline
    (Abstracción) estado = conjunto de estados reales\newline
    (Abstracción) acción = combinacion compleja de acciones reales\newline
   
    \hspace*{1em} e.g., “Arad $\rightarrow$ Zerind” representa un conjunto complejo de posibles rutas, desvíos, parades de descanso, etc. \newline
    Para garantizar la realización, \textcolor{red}{cualquier} estado real "en Arad" debe llegar a un estado real "en Zerind"\newline
    
    (Abstracción) solución = \newline
    \hspace*{1em} conjunto de caminos reales que son soluciones en el mundo real
    
    ¡Cada acción abstracta debería ser "más fácil" que el problema original!    
    
\end{frame}
