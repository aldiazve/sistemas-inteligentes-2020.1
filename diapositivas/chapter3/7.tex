\begin{frame}{Tipos de problemas.}
\textcolor{green}{Determinístico, completamente observable} $\Rightarrow$ \textcolor{blue}{problema de único estado} \newline
\blank{1cm}El agente sabe exactamente en qué estado estará; la solución es una secuencia.
\newline
\textcolor{green}{No observable} $\Rightarrow$ \textcolor{blue}{estado conforme} \newline
\blank{1cm}El agente puede no tener idea de dónde esta; la solucion (si existe) es una secuencia.
\newline
\textcolor{green}{No determinístico y/o parcialmente observable} $\Rightarrow$ \textcolor{blue}{problema continuos} \newline
\blank{1cm}las percepciones proveen información \textcolor{red}{nueva} sobre el estado actual\newline
\blank{1cm}la solución es un \textcolor{blue}{plan de contingencia} o \textcolor{blue}{una póliza}\newline
\blank{1cm}con frecuencia una búsqueda \textcolor{red}{intercalada}, ejecución.\newlin
\newline
\textcolor{green}{Estado espacial desconocido} $\Rightarrow$ \textcolor{blue}{problema de exploración} (“en línea”)
\end{frame}
