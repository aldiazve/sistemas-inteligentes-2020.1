% página 27
\begin{frame}{Resumen}{}
Los \emphblue{agentes} interactúan con los \emphblue{ambientes} a través de \emphblue{sensores} y \emphblue{actuadores}\\

La \emphblue{función del agente} describe que hace el agente en todo momento\\

La \emphblue{medida de desempeño} evalua la secuencia del ambiente\\

Un \emphblue{agente perfectamente racional} maximiza el desempeño esperado\\

Los \emphblue{programas del agente} implementan (algunas) funciones de agente\\

Las descripciones \emphblue{MAAS} describen los entornos de trabajo\\

Los ambientes son categorizados entre varias dimensiones:\\
    \emphblue{Observable}? \emphblue{Determinístico}? \emphblue{episódico}? \emphblue{estático}? \emphblue{discreto}? \emphblue{agente único}?\\
   
Existen múltiples arquitecturas de agentes básicos:\\
    \emphblue{reactivos}, \emphblue{reactivos con estado}, \emphblue{basados en objetivos}, \emphblue{basados en utilidad}
\end{frame}