%Capitulo 2, presentación 21
\begin{frame}[fragile]{Ejemplo}
 \begin{center}
  \begin{tabular}{|l|} 
   \hline
   	La \textbf{función} Reflex-Vacuum-Agent(\textit{[ubicación,estado]}) \textbf{retorna} una acción \\
    \hspace{1em} \textbf{sí} \textit{estado = Sucio} \textbf{entonces retorna} \textit{Aspirar} \\
    \hspace{1em} \textbf{entonces sí} \textit{ubicación = A} \textbf{entonces retorna} \textit{Derecha} \\
    \hspace{1em} \textbf{entonces sí} \textit{ubicación = B} \textbf{entonces retorna} \textit{Izquierda} \\
   \hline
  \end{tabular}
  {\scriptsize 
	\begin{verbatim}
	(setq joe (hacer-agente :nombre 'joe :cuerpo (hacer-cuerpo-agente)
                                 :programa (hacer-programa-reflex-vacuum-agent))
    (defun hacer-programa-reflex-vacuum-agent()
      #'(lambda (percepción)
          (sea ((locación (primera percepción)) (estado (segunda percepción)))
            (cond ((eq estado 'sucio) 'Aspirar)
                  ((eq ubicación 'A) 'Derecha)
                  ((eq ubicación 'B) 'Izquierda)))))
  \end{verbatim}
  }
 \end{center}
\end{frame}