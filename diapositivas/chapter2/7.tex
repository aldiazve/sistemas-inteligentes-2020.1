\begin{frame}{Racionalidad}
\textcolor{blue}{Medida de desempeño} fija que evalúa la secuencia del ambiente
\begin{itemize}
  \item un punto por cuadrado limpiado en un tiempo $T$ ?
  \item un punto por cuadrado limpio por paso temporal, menos uno por movimiento?
   \item penalizar por cantidad de cuadrados sucios $>k$ ?
\end{itemize}
Un \textcolor{red}{agente racional} escoge cualquier acción que maximice el valor \textcolor{red}{esperado} de la medida de desempeño \textcolor{red}{dada la secuencia percibida hasta el momento.}

\begin{itemize}
   \item Racional $\neq$  omnipresente.
   \begin{itemize}
     \item las percepciones pueden no ofrecer toda la información relevante.
   \end{itemize}
    \item Racional $\neq$  clarividente.
   \begin{itemize}
     \item los resultados de las acciones pueden no ser los esperados.
   \end{itemize}
    \item por lo tanto, racional $\neq$ éxito.
     \item Racional $\Rightarrow$ explotación, aprendizaje, autonomía.
\end{itemize}
\end{frame}
