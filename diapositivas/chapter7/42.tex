%chap 7 pag 42
\definecolor{Purple}{RGB}{153,0,153}
\definecolor{Green}{RGB}{0,100,0}
\definecolor{DarkRed}{RGB}{128,0,0}

\begin{frame}{Encadenamiento hacia adelante y hacia atrás}
    \textcolor{blue}{Cláusula de Horn} (restringida)\\
    
    \hspace{0.5cm}KB = \textcolor{DarkRed}{Conjunción} de las
    \textcolor{DarkRed}{cláusulas deHorn}\\
    
    Cláusula de Horn =
    \begin{itemize}
        \item Símbolo de proposición; o
        \item (Conjunción de símbolos) \textcolor{Purple}{$\Rightarrow$} símbolo
    \end{itemize}{}
    \textcolor{blue}{Modus Ponens}(para la cláusula de Horn): Completa para KBs de
    Horn.
    
    \begin{equation*}
        \textcolor{Purple}{\frac{\alpha_1, \cdots, \alpha_n \hspace{0.5cm} \alpha_1
        \wedge \cdots \wedge \alpha_n \Rightarrow \beta}{\beta}}
    \end{equation*}{}
    
    Se puede usar con \textcolor{blue}{encadenamiento hacia adelante} o \textcolor{blue}{hacia atrás}.
    Estos algoritmos son muy naturales y se ejecutan en tiempo \textcolor{DarkRed}{lineal}.
\end{frame}{}
