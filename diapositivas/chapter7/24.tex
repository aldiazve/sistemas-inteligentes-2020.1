%chap 7 pag  24
\definecolor{Purple}{RGB}{153,0,153}
\definecolor{Green}{RGB}{0,100,0}

\begin{frame}{Modelos}
	Los lógicos suelen pensar en términos de \textcolor{blue}{modelos},
	que son mundos formalmente estructurados con respecto a los cuales se
	puede evaluar la verdad.\\[0.3cm]

	Decimos que \textcolor{Purple}{$m$} \textcolor{Green}{es un modelo de} 
	una sentencia \textcolor{Purple}{$\alpha$} si \textcolor{Purple}{$\alpha$} es
	verdadero en \textcolor{Purple}{$m$}\\[0.3cm]

	\textcolor{Purple}{$M(\alpha)$} es el conjunto de todos los modelos de
	\textcolor{Purple}{$\alpha$}\\[0.3cm]

	Entonces \textcolor{Purple}{$KB \models \alpha$} si y solo si
	\textcolor{Purple}{$M(KB) \subseteq M(\alpha)$}\\[0.3cm]

	Por ejemplo:\\[0.3cm]

	\textcolor{Purple}{$KB =$} Gigantes ganaron y Rojos ganaron\\
	\textcolor{Purple}{$\alpha =$} Gigantes ganaron
\end{frame}{}
