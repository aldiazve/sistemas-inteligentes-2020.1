%CAPITULO 6
%pagina 23
\begin{frame}{La Implicación}
La \textcolor{blue}{Implicación} significa que una cosa
\textcolor{red}{se sigue de} otra:\\
\textcolor{purple}{KB $\models \alpha$ }\\
La Base de conocimiento \textcolor{purple}{KB} implica la sentencia \textcolor{purple}{ $\alpha$ } \\ 
\hspace{1cm} Si y solo sí \\
\textcolor{purple}{$\alpha$} es verdadera en todos los mundos donde \textcolor{purple}{KB} es verdera. \\
\vspace{0.1cm} Por ejemplo, la KB contiene \textit{los Gigantes ganaron"} y \textit{"los Rojos ganaron"} implica  \textit{"O ganaron los Gigantes o ganaron los Rojos" }\\ 
\vspace{0.1cm}
Por ejemplo, \textcolor{purple}{$x + y =4$} implica \textcolor{purple}{$4=x + y$}\\
\vspace{0.1cm}
La implicación es una relación entre oraciones (por ejemplo, \textcolor{red}{sintaxis}) que se basa en la \textcolor{red}{semántica}. \\
\vspace{0.8cm} \textcolor{red}{Nota:} la \textcolor{red}{sintaxis} del proceso del cerebro (de algún tipo)
\end{frame}