%CAPITULO 6
%pagina 71
\begin{frame}{Resumen}
\scriptsize{
Los agentes logicos aplican \textcolor{blue}{inferencia} a una \textcolor{blue}{base de conocimiento} para derivar nueva información y tomar decisiones.\\
\vspace{0.3cm}
Conceptos Basicos de lógica:
\begin{enumerate}
\item \textcolor{blue}{Sintaxis}: Estructura formal de las  \textcolor{blue}{oraciones}
\item \textcolor{blue}{Semántica}: \textcolor{blue}{Verdad} de las sentencias de los \textcolor{blue}{modelos} wrt.
\item \textcolor{blue}{Implicación}: Verdad necesaria de una oración dada otra.
\item \textcolor{blue}{Inferencia}: Derivación de oraciones apartir de otras oraciones.
\item \textcolor{blue}{Solvencia}: Las derivaciones producen solo oraciones implicadas
\item \textcolor{blue}{Completitud}: Las derivaciones pueden producir \textcolor{blue}{todas} las oraciones implicadas
\end{enumerate}
El mundo Wumpus requiere la capacidad de representar información parcial y negada, razón por casos, etc.\\ 
\vspace{0.3cm}
El encadenamiento hacia adelante y hacia atrás es de tiempo lineal, para las cláusulas Horn la resolución está completa para la lógica proposicional.\\ 
\vspace{0.3cm}
\hspace{1cm} La lógica proposicional carece de poder expresivo.
}
\end{frame}