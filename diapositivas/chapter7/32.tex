\begin{frame}{Lógica proposicional: Sintaxis}
    
    \textcolor{red}{Observable??} No 
    La lógica proposicional es la lógica más simple — ilustra ideas básicas
    \bigskip
    
    Los símbolos de proposición {\fontfamily{qcr}\selectfont
    \textcolor{purple}{P\textsubscript{1}},\textcolor{purple}{P\textsubscript{2}}
    }, etc. son oraciones
    \bigskip
    
    Si {\fontfamily{qcr}\selectfont
    \textcolor{purple}{S}} es una oración, {\fontfamily{qcr}\selectfont
    \textcolor{purple}{$\neg$S}} es una oración (\textcolor{blue}{negación})
    \bigskip
    
    Si {\fontfamily{qcr}\selectfont
    \textcolor{purple}{S\textsubscript{1}}} y {\fontfamily{qcr}\selectfont \textcolor{purple}{S\textsubscript{2}}
    }son oraciones, {\fontfamily{qcr}\selectfont
    \textcolor{purple}{S\textsubscript{1}$\land$S\textsubscript{2}}} es una oración (\textcolor{blue}{conjunción})
    \bigskip
    
    Si {\fontfamily{qcr}\selectfont
    \textcolor{purple}{S\textsubscript{1}}} y {\fontfamily{qcr}\selectfont \textcolor{purple}{S\textsubscript{2}}
    }son oraciones, {\fontfamily{qcr}\selectfont
    \textcolor{purple}{S\textsubscript{1}$\lor$S\textsubscript{2}}} es una oración (\textcolor{blue}{disyunción})
    \bigskip
    
    Si {\fontfamily{qcr}\selectfont
    \textcolor{purple}{S\textsubscript{1}}} y {\fontfamily{qcr}\selectfont \textcolor{purple}{S\textsubscript{2}}
    }son oraciones, {\fontfamily{qcr}\selectfont
    \textcolor{purple}{S\textsubscript{1}$\Rightarrow$S\textsubscript{2}}} es una oración (\textcolor{blue}{implication})
    \bigskip
    
    Si {\fontfamily{qcr}\selectfont
    \textcolor{purple}{S\textsubscript{1}}} y {\fontfamily{qcr}\selectfont \textcolor{purple}{S\textsubscript{2}}
    }son oraciones, {\fontfamily{qcr}\selectfont
    \textcolor{purple}{S\textsubscript{1}$\Leftrightarrow$S\textsubscript{2}}} es una oración (\textcolor{blue}{biconditional})
    \bigskip
    
\end{frame}