%Capitulo 7, presentación 69
\definecolor{Red}{RGB}{100,10,10}
\definecolor{DarkGreen}{RGB}{10,100,10}
\definecolor{DarkBlue}{RGB}{10,10,100}
\definecolor{DarkPurple}{RGB}{100,0,110}
\begin{frame}{Algoritmo de resolución}
	Prueba por contradicción, es decir, demostrar que no se satisface \textcolor{DarkPurple}{$KB \land \neg\alpha$}
	\begin{center}
		\begin{tabular}{|l|}
		\hline
		{\footnotesize
			\textcolor{DarkBlue}{función} \textcolor{Red}{\textsc{Resolución-PL}}								(\textcolor{DarkGreen}{KB,{$\alpha$}}) \textcolor{DarkBlue}{retorna} 								\textcolor{DarkGreen}{\textit{verdadero}} o \textcolor{DarkGreen}{\emph{falso}}
		}
		\\
		{\footnotesize
			\qquad 
			\textcolor{DarkBlue}{entradas}: \textcolor{DarkGreen}{KB}, el conocimiento base, 					una	sentencia en lógica proposicional 
		}
		\\	
		{\footnotesize
			\qquad
			\quad\quad\quad\quad\quad
			\textcolor{DarkGreen}{$\alpha$}, la consulta,una sentencia en lógica proposicional 
		}
		\\
		{\footnotesize
			\qquad
			\textcolor{DarkGreen}{\textit{clausulas}} {$\leftarrow$} el conjunto de clausulas en la 				representación CNF de \textcolor{DarkGreen}{KB}{$ \land \neg\alpha$}
		}
		\\
		{\footnotesize
			\qquad
			\textcolor{DarkGreen}{\textit{nuevo}} {$\leftarrow$} {$\{\}$}
		}
		\\
		{\footnotesize
			\qquad
			\textcolor{DarkBlue}{búcle hacer}	
		}
    	\\
    	{\footnotesize
			\qquad\qquad
			\textcolor{DarkBlue}{por cada} {$C_i, C_j$} \textcolor{DarkBlue}{en} 								\textcolor{DarkGreen}{clausulas} \textcolor{DarkBlue}{hacer}	
		}
    	\\
    	{\footnotesize
			\qquad\qquad\qquad
			\textcolor{DarkGreen}{\textit{solventes}} {$\leftarrow$} \textsc{Resolver-PL}({$C_i, C_j				$})
		}
    	\\
    	{\footnotesize
			\qquad\qquad\qquad
			\textcolor{DarkBlue}{si} \textcolor{DarkGreen}{solventes} contienen la clausula vacía 				\textcolor{DarkBlue}{entonces retornar} \textcolor{DarkGreen}{\textit{verdadero}}		
		}
    	\\
    	{\footnotesize
			\qquad\qquad\qquad
			\textcolor{DarkGreen}{\textit{nuevo}} {$\leftarrow$} \textcolor{DarkGreen}							{\textit{nuevo}} {$\cup$} \textcolor{DarkGreen}{\textit{solventes}}
		}
    	\\
    	{\footnotesize
			\qquad\qquad
			\textcolor{DarkBlue}{si} \textcolor{DarkGreen}{\textit{nuevo}} {$\subseteq$}						\textcolor{DarkGreen}{\textit{clausulas}} \textcolor{DarkBlue}{entonces retornar} 					\textcolor{DarkGreen}{\textit{falso}}		
		}
    	\\
    	{\footnotesize
			\qquad\qquad
			\textcolor{DarkGreen}{\textit{clausulas}} {$\leftarrow$} \textcolor{DarkGreen}						{\textit{clausulas}} {$\cup$} \textcolor{DarkGreen}{\textit{nuevo}}
		}
    	\\
   		\hline
   		
  		\end{tabular}
	\end{center}
\end{frame}